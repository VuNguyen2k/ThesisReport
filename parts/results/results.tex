\newcommand{\reimg}[1]{parts/results/img/#1}
\setupfont{13pt}

\chapter{Thực nghiệm và kết quả}\label{chapter::results}

	Trong chương này, chúng tôi sẽ trình bày về quá trình xử lý dữ liệu, huấn luyện mô hình cũng như một số kết quả mà chúng tôi thu được với mô hình của mình. Các mô hình của chúng tôi sẽ được đem đi thử nghiệm so sánh với một số mô hình đã có và các mô hình được các ứng dụng công nghiệp mã nguồn đóng cũng như mã nguồn mở trên bộ dữ liệu tiếng Việt. Bộ dữ liệu mở rộng đa ngôn ngữ được chúng tôi thử nghiệm để kiểm thử khả năng tổng quát hóa trên nhiều loại ngôn ngữ khác nhau của mô hình.

\section{Chuẩn bị dữ liệu}\label{section::results::data_preparation}
	
	Để chuẩn bị cho quá trình huấn luyện mô hình, chúng tôi cần phải bắt đầu từ khâu chuẩn bị dữ liệu. Dữ liệu được chúng tôi sử dụng ở trong quá trình huấn luyện là dữ liệu đã được tiền xử lý, quá trình tiền xử lý này sẽ trộn nhiễu với âm sạch theo một tỉ lệ cho trước, tỉ lệ mà chúng tôi đang nói đến là \textbf{Signal To Noise Rate (SNR)}. SNR được định nghĩa như sau
	
		\begin{equation}
			\name{SNR}(x, y) = 10\log_{10}\bigg(\frac{\norm{x}^2_2}{\norm{y}^2_2}\bigg),
		\label{re::snr_formula}
		\end{equation}
	
	\noindent với $x, y \in \mathbb{R}^n$. Bằng công thức ở trên, chúng tôi đã tạo ra hai loại nhiễu trong dữ liệu của mình, đó là \textbf{nhiễu tương quan (correlated noise)} và \textbf{nhiễu không tương quan (uncorrelated noise)}. Điều khác biệt duy nhất ở hai loại nhiễu này đó chính là sự biến đổi biên độ của nhiễu có tương quan với giọng nói hay không. Để làm rõ điều này ta lấy một ví dụ. \figref{re::clean_sound_1} và \figref{re::clean_sound_2} bên dưới chính là giọng nói sạch không có nhiễu, ta mong muốn pha một nhiễu vào trong âm thanh này sao cho nhiễu đó có tồn tại tại sự tương quan về mặt biên độ đối với giọng nói hay khi biên độ giọng nói tăng lên, nhiễu cũng sẽ tăng lên. Và ngược lại khi biên độ giọng nói giảm thì biên độ của nhiễu cũng sẽ bị giảm theo. Hai đại lượng này tỉ lệ với nhau nhưng vẫn không làm mất đi bản chất của nhiễu ban đầu. Kết quả của việc pha nhiễu này được thể hiện ở \figref{re::mix_corr_noise_s} và \figref{re::mix_corr_noise_w}.
	
		\begin{figure}[h]
			\centering
			\begin{subfigure}{.35\textwidth}
				\centering
				\includegraphics[width=50mm]{\reimg{corr_x_s.png}}
				\caption{Âm thanh sạch}
				\label{re::clean_sound_1}
			\end{subfigure}%
			\begin{subfigure}{.35\textwidth}
				\centering
				\includegraphics[width=50mm]{\reimg{corr_n_s.png}}
				\caption{Âm nhiễu tương quan}
			\end{subfigure}%
			\begin{subfigure}{.35\textwidth}
				\centering
				\includegraphics[width=50mm]{\reimg{corr_y_s.png}}
				\caption{Âm thanh đã pha nhiễu}
			\end{subfigure}
			\caption{Kết quả \spectrogram{} của việc trộn nhiễu tương quan}
		\label{re::mix_corr_noise_s}
		\end{figure}
	
		\begin{figure}[h]
			\centering
			\begin{subfigure}{.35\textwidth}
				\centering
				\includegraphics[width=50mm]{\reimg{corr_x_w.png}}
				\caption{Âm thanh sạch}
			\end{subfigure}%
			\begin{subfigure}{.35\textwidth}
				\centering
				\includegraphics[width=50mm]{\reimg{corr_n_w.png}}
				\caption{Âm nhiễu tương quan}
			\end{subfigure}%
			\begin{subfigure}{.35\textwidth}
				\centering
				\includegraphics[width=50mm]{\reimg{corr_y_w.png}}
				\caption{Âm thanh đã pha nhiễu}
			\end{subfigure}
			\caption{Kết quả \waveform{} của việc trộn nhiễu tương quan}
			\label{re::mix_corr_noise_w}
		\end{figure}
	
	
		\begin{figure}[h]
			\centering
			\begin{subfigure}{.35\textwidth}
				\centering
				\includegraphics[width=50mm]{\reimg{uncorr_x_s.png}}
				\caption{Âm thanh sạch}
				\label{re::clean_sound_2}
			\end{subfigure}%
			\begin{subfigure}{.35\textwidth}
				\centering
				\includegraphics[width=50mm]{\reimg{uncorr_n_s.png}}
				\caption{Âm nhiễu không tương quan}
			\end{subfigure}%
			\begin{subfigure}{.35\textwidth}
				\centering
				\includegraphics[width=50mm]{\reimg{uncorr_y_s.png}}
				\caption{Âm thanh đã pha nhiễu}
			\end{subfigure}
			\caption{Kết quả \spectrogram{} của việc trộn nhiễu không tương quan}
		\label{re::mix_uncorr_noise_s}
		\end{figure}
	
		\begin{figure}[h]
			\centering
			\begin{subfigure}{.35\textwidth}
				\centering
				\includegraphics[width=50mm]{\reimg{uncorr_x_w.png}}
				\caption{Âm thanh sạch}
			\end{subfigure}%
			\begin{subfigure}{.35\textwidth}
				\centering
				\includegraphics[width=50mm]{\reimg{uncorr_n_w.png}}
				\caption{Âm nhiễu không tương quan}
			\end{subfigure}%
			\begin{subfigure}{.35\textwidth}
				\centering
				\includegraphics[width=50mm]{\reimg{uncorr_y_w.png}}
				\caption{Âm thanh đã pha nhiễu}
			\end{subfigure}
			\caption{Kết quả \waveform{} của việc trộn nhiễu không tương quan}
			\label{re::mix_uncorr_noise_w}
		\end{figure}
	
	Trái với nhiễu tương quan, nhiễu không tương quan không có sự tương đồng như vậy. Nhiễu ở trong trường hợp này chỉ được nhân thêm một hằng số sao cho sự giá trị \name{SNR} của nhiễu và âm sạch nằm ở một mức mong muốn.  \figref{re::mix_uncorr_noise_s} thể hiện \spectrogram{} nhiễu trong trường hợp không có sự tương quan này. Về mặt toán học, cả hai loại nhiễu đều được tính toán dựa trên chung một chỉ số \name{SNR} cho trước và sử dụng công thức \formularef{re::snr_formula}. Ở trong trường hợp nhiễu tương quan, công thức của \name{SNR} được thay đổi để phù hợp với sự biến đổi của giọng nói
	
		\begin{align*}
			n'_1(t) = n(t) \sqrt{\frac{x(t)^2}{10^{\alpha / 10}}}. \numberedeq
			\label{re::corr_noise_formula}
		\end{align*}
	
	Bằng cách chuẩn hóa lại giá trị nhiễu mong muốn được tính từ $\sqrt{x(t)^2/10^{\alpha / 10}}$, chúng tôi nhân với giá trị nhiễu hiện tại $n(t)$, từ đó thu được sự biến đổi của nhiễu theo giọng nói như mong muốn. Tuy cách làm như thế này không đảm bảo được giá trị \name{SNR} được tính trong công thức ban đầu là đúng chính xác như \name{SNR} kì vọng $\alpha$, vì trong trường hợp này, nếu ta chỉ đảm bảo được \name{SNR} trong công thức tại một thời điểm bất kì luôn là giá trị mong đợi $\alpha$ thì điều đó cũng không có nghĩa trong công thức ban đầu cũng như vậy. Nhưng trong nhiễu không tương quan, giá trị \name{SNR} tính theo công thức ban đầu hoàn toàn có thể được đảm bảo đúng với giá trị mong đợi. Nhiễu không tương quan được định nghĩa như sau
	
		\begin{align*}
			n'_2(t) = \frac{n(t)}{\norm{n}_2} \frac{\norm{x}_2}{\sqrt{10^{\alpha / 10}}}. \numberedeq
			\label{re::uncorr_noise_formula}
		\end{align*}
	
	Việc \name{SNR} được cố định lại thể hiện chính xác trong công thức trên. Ở vế $\norm{x}_2 / \sqrt{10^{\alpha / 10}}$, thay vì sử dụng \name{SNR} trên từng thời điểm chúng tôi triển khai lại đúng với chuẩn Euclid của $x(t)$. Từ đó tính ra được chuẩn Euclid của biên độ nhiễu $n(t)$ mới là $\norm{n'}_2 = \norm{x}_2 / \sqrt{10^{\alpha / 10}}$. 
	
	Bằng cách nhân mỗi đại lượng tại $x(t)$ lên thêm một giá trị $\norm{n'}_2 / \norm{n}_2$, chúng tôi đang nhân giá trị chuẩn Euclid của $n(t)$ cho một hằng số để thỏa mãn giá trị tương ứng với \name{SNR} dược tính toán từ $x(t)$. Từ đó \name{SNR} của nhiễu so với giọng nói sạch theo cách pha nhiễu này đảm bảo được \name{SNR} được tính ở công thức gốc dùng âm thanh sạch và nhiễu đã xử lý sẽ đúng với giá trị \name{SNR} mong đợi $\alpha$.
	
	Về dữ liệu, trong việc sử dụng để huấn luyện mô hình, dữ liệu sạch của chúng tôi gồm có hai loại, dữ liệu tiếng Anh và dữ liệu tiếng Việt. Về dữ liệu tiếng Anh, theo tìm hiểu của chúng tôi, các công trình nghiên cứu được công bố trong các cuộc thi giảm nhiễu âm thanh được công bố khá nhiều, do vậy chúng tôi đã sử dụng một phần dữ liệu trong các cuộc thi này về để sử dụng như nguồn âm thanh sạch của mình. \tableref{re::eng_datasets} mô tả một số bộ dữ liệu tiếng Anh mà chúng tôi sử dụng.
	
		\begin{table}[h]
			\centering
			\begin{tabular}{l c c c c}
				\hline
				\textbf{Dataset} 			& \textbf{Kích thước}	& \textbf{Mục tiêu}	& \textbf{Số loại nhiễu}& \textbf{Giọng nói sạch} \\
				\hline
				DNS \cite{dns}				& 1 TB					& 	Giảm nhiễu		& 150					& 11350 \\
				Edinburgh Data \cite{edata}	& 5 GB					& 	Giảm nhiễu,
																		Text to Speech	& 2						& 28 \\
				Edinburgh Data \cite{edata}	& 10 GB					& 	Giảm nhiễu,
																		Text to Speech	& 2						& 56 \\
				CHIME \cite{chime}			& 115 GB				& 	Speech to Text,
																		tách giọng nói	& 0						& 32 \\
				DEMAND \cite{demand}			& 7.4 GB				&	Noise Dataset	& 18					& 0 \\
				\hline
			\end{tabular}
		\caption{Các tập dữ liệu tiếng Anh được sử dụng trong luận văn}
		\label{re::eng_datasets}
		\end{table}
	
	Bên cạnh đó, chúng tôi cũng đã thu thập được một số bộ dữ liệu tiếng Việt. Tuy mục đích của chúng không phải dùng để giảm nhiễu trong âm thanh mà là để dùng cho nhận diện giọng nói sang chữ viết nên chúng tôi phải tự lọc lại dữ liệu. Vì nhiễu đã được pha trước vào dữ liệu và thứ mà chúng tôi cần là một giọng nói sạch, cũng như lựa ra những bộ dữ liệu có chất lượng tốt để dùng cho việc huấn luyện của mình. 
	
	\tableref{re::vie_datasets} dưới đây là mô tả về bộ dữ liệu tiếng Việt mà chúng tôi sẽ sử dụng. Để đảm bảo tính công bằng giữa các mô hình đã có và mô hình của mình, mô hình chúng tôi sẽ chỉ được huấn luyện trên bộ dữ liệu tiếng Anh. Bộ dữ liệu tiếng Việt sẽ được sử dụng như bộ kiểm thử chất lượng của mô hình.
	
		\begin{table}[h]
			\centering
			\begin{tabular}{l c c c}
				\hline
				\textbf{Dataset} 		& \textbf{Kích thước}	& \textbf{Mục tiêu}	& \textbf{Số file âm thanh} \\
				\hline
				FPT Open Dataset		& 1.6 GB				& 	Speech to Text 	& 15049 \\
				VIVOS \cite{vivos}		& 1.4 GB				& 	Speech to Text 	& 12426 \\
				VLSP 2019				& 23 GB					& 	Speech to Text 	& 160000 \\
				VIET-TTS				& 5 GB					& 	Speech to Text, Text to Speech & 22884 \\
				\hline
			\end{tabular}
			\caption{Các tập dữ liệu tiếng Việt được sử dụng trong luận văn}
			\label{re::vie_datasets}
		\end{table}
	
	%Với lượng dữ liệu trên, chúng tôi tiến hành huấn luyện mô hình, với mỗi mẫu giọng nói sạch chúng tôi sẽ chọn ngẫu nhiên $k$ mẫu nhiễu (lấy mẫu ngẫu nhiên có hoàn lại) và cắt lấy một đoạn bất kì trong mẫu nhiễu đó trộn vào giọng nói sạch. Theo thực nghiệm của chúng tôi, với $k = 4$ và \name{SNR} nằm trong khoảng $[-6, 20]$ là phù hợp cho mô hình mà chúng tôi đề xuất.
	Bộ dữ liệu mà chúng tôi dùng để huấn luyện mô hình sẽ bao gồm các âm sạch được lấy ngẫu nhiên 4 lần để lai tạo với các âm nhiễu khác nhau, tỉ lệ chia nhiễu tương đồng và nhiễu không tương đồng là 3:7, với \name{SNR} được lấy ngẫu nhiên trong khoảng $[-6, 20]$.
	
	Tuy nhiên trong quá trình hiện thực mô hình, chúng tôi gặp phải một số vấn đề liên quan tới biên độ của giọng nói đầu vào từ microphone của người dùng. Với giả định khi huấn luyện mô hình rằng microphone của người dùng sẽ luôn nằm ở đúng vị trí của miệng lúc nói, do vậy các âm thanh chúng tôi dùng để huấn luyện luôn được chuẩn hóa về biên độ lớn nhất là $1.0$, nhưng thực tế lại không phải vậy.
	
		\begin{figure}[h]
			\centering
			\includegraphics[width=120mm]{\reimg{mic_distance.png}}
			\caption{Luôn tồn tại khoảng cách từ miệng người nói tới microphone}
		\label{re::mic_distance}
		\end{figure}
	
	\figref{re::mic_distance} thể hiện cho việc này, với sự xuất hiện của khoảng cách từ miệng người nói tới microphone như vậy, biên độ của âm thanh lúc này sẽ không thể được đảm bảo như giả định của chúng tôi cũng như khi người dùng thay đổi vị trí của microphone, điều này cũng gây ra sự biến đổi trong biên độ. Nhận thấy điều này, bằng các thực nghiệm, chúng tôi xác định đối với microphone của chúng tôi khoảng hoạt động của biên độ thường nằm trong khoảng $[0.2, 0.6]$ và phải để rất gần (gần như là trước miệng) để biên độ được microphone ghi nhận chạm tới mức $1.0$. 
	
	Vậy nên để có thể giúp cho mô hình của chúng tôi có thể hoạt động trong các môi trường như vậy, chúng tôi giả lập lại tính huống trên bằng cách thay đổi một cách ngẫu nhiên trong một giá trị $\delta \in [0.1, 1.0]$. Và giọng nói chúng tôi dùng để huấn luyện mô hình dùng để sẽ được nhân với hệ số $\delta$ này. 
	
	Phát hiện này của chúng tôi khá trễ, vào giai đoạn mà chúng tôi đang hiện thực ứng dụng để áp dụng mô hình nên các mô hình phiên bản trước Phiên bản 3, sẽ không được kiểm thử trên các âm chưa được chuẩn hóa về biên độ có giá trị là $1.0$. Chỉ có phiên bản cuối và các mô hình cạnh tranh sẽ được kiểm thử trên tất cả biên độ.
	
	Sau khi đã chuẩn bị hoàn tất dữ liệu, chúng tôi bắt đầu khâu chuẩn bị dữ liệu kiểm thử. Khác với dữ liệu huấn luyện, dữ liệu kiểm thử chỉ được chúng tôi sử dụng trên tiếng Việt được lấy từ bộ kiểm thử của tập dữ liệu VIVOS \cite{vivos} thay vì sử dụng tiếng Anh và có thể có nhiều loại nhiễu không tương đồng cùng tồn tại trong cùng một đoạn âm thanh để giả lập lại môi trường xung quanh người dùng. Bộ dữ liệu kiểm thử được chúng tôi chia thành năm mức biên độ $A$ là $\{0.2, 0.4, 0.6, 0.8, 1.0\}$, và tương ứng cho mỗi mức biên độ sẽ có sáu mức \name{SNR} là $\{-5, 0, 5, 10, 15, 20\}$. Mỗi cặp giá trị $A$ và \name{SNR} này sẽ được tạo một bộ dữ liệu nhỏ có 400 mẫu. 
	
	Như vậy với bộ dữ liệu kiểm thử trên, chúng tôi sẽ tiến hành thử nghiệm mô hình trên tất cả 12000 mẫu dữ liệu có số lượng nhiễu được lấy ngẫu nhiên từ một đến bốn loại (có thể trùng) với các mức biên độ và \name{SNR} tương ứng. Chúng tôi chạy thử các metrics của mình trên bộ kiểm thử này và thu được kết quả như \tableref{re::baseline}.
	
		\begin{table}[h]
			\centering
			\begin{tabular}{l c c c c c}
				\hline
				\multicolumn{1}{c}{\textbf{Bộ kiểm thử}}	& \textbf{STOI}	& \textbf{NBPESQ}	& \textbf{WBPESQ}	& \textbf{SIG}	& \textbf{BAK} \\
				\hline
				$\name{SNR}=-5$        & 0.64 (0.11)   & 1.36 (0.27)   & 1.10 (0.12)   & 3.66 (0.34)   & 0.98 (0.71) \\
				$\name{SNR}=0$         & 0.75 (0.09)   & 1.54 (0.36)   & 1.16 (0.19)   & 3.89 (0.27)   & 1.21 (0.73) \\
				$\name{SNR}=5$         & 0.84 (0.08)   & 1.78 (0.41)   & 1.29 (0.28)   & 4.15 (0.20)   & 1.63 (0.76) \\
				$\name{SNR}=10$        & 0.90 (0.07)   & 2.11 (0.46)   & 1.54 (0.37)   & 4.32 (0.15)   & 2.20 (0.69) \\
				$\name{SNR}=15$        & 0.94 (0.05)   & 2.57 (0.50)   & 1.97 (0.48)   & 4.39 (0.15)   & 2.75 (0.60) \\
				$\name{SNR}=20$        & 0.96 (0.04)   & 3.08 (0.48)   & 2.51 (0.51)   & 4.38 (0.19)   & 3.25 (0.46) \\
				\hline
			\end{tabular}
			\caption{Metric trung bình trên từng bộ kiểm thử tiếng Việt ứng với các $\name{SNR}$ khác nhau}
			\label{re::baseline}
		\end{table}
	
	Để có thể kiểm thử kĩ hơn nữa khả năng lọc nhiễu của mô hình, chúng tôi tiến hành tạo thêm một bộ kiểm thử khác. Trong bộ trong bộ này, chúng tôi sử dụng giọng nói được lấy từ sách nói ở thư viện mở Librivox\footnote{\url{https://librivox.org/}} của sáu thứ tiếng khác nhau bao gồm: \textit{tiếng Nhật}, \textit{tiếng Pháp}, \textit{tiếng Nga}, \textit{tiếng Đức}, \textit{tiếng Bồ Đào Nha} và \textit{tiếng Trung Quốc}. Vì đã xác định việc thay đổi biên độ của âm đầu vào không làm ảnh hưởng nhiều tới chất lượng lọc nhiễu của mô hình (thông qua các số liệu được trình bày trong phần kết quả), chúng tôi thực hiện lai tạo giọng nói trên cũng với sáu mức \name{SNR}. Các âm nhiễu có thể xuất hiện cùng lúc bên trong giọng nói nhưng với biên độ được lấy ngẫu nhiên theo phân phối đều trong khoảng $[0.1, 1.0]$. Chúng tôi cũng tiến hành chạy kiểm thử các kết quả metrics trên bộ kiểm thử đa ngôn ngữ này và thu được kết quả như trong \tableref{re::multilingual_baseline}.

\section{Một số kết quả của mô hình lọc nhiễu}
	
	Bằng các phương pháp lai tạo, sử dụng các mô hình và hàm mất mát đã được đề xuất ở trên, chúng tôi đã huấn luyện ra được các mô hình với kết quả khá khả quan. Mô hình được chúng tôi sử dụng chính ở trong ứng dụng của mình là mô hình được đề xuất ở Phiên bản 3 với bộ nén dữ liệu sử dụng tích chập. Đó cũng chính là mô hình mà chúng tôi sẽ đem đi so sánh với mô hình đối thủ (mô hình có điểm metrics cao nhất trong các mô hình được chúng tôi tham khảo). Các metrics được chúng tôi sử dụng bao gồm STOI, PESQ và DNSMOS (chúng tôi chỉ xét SIG và BAK, OVR sẽ không được xét vì không mang nhiều thông tin về chất lượng mà chúng tôi muốn đánh giá).
	
	Đầu tiên để có thể tiến hành so sánh chất lượng của mô hình chúng tôi với một số mô hình đã có được chúng tôi lấy từ các bài báo bao gồm FullSubNet \cite{fullsubnet}, DCCRN \cite{dccrn} và DTLN \cite{dtln}. Trong số ba mô hình này, chúng tôi sẽ chọn lấy mô hình có điểm metrics cao nhất từ đó đem chúng đi so sánh với mô hình của mình. Điểm metrics của bộ kiểm thử và các mô hình đối thủ được chúng tôi trình bày trong \tableref{re::competitors_metrics}. Như các bảng metrics cũng đã cho thấy mô hình DTLN đạt các chỉ số metrics cao nhất so với hai mô hình còn lại do đó các chỉ số của mô hình DTLN này sẽ được đem đi so sánh với mô hình của chúng tôi.
	
		\begin{table}[h]
			\centering
			\begin{tabular}{l c c c c c c}
				\hline
				\multicolumn{1}{c}{\textbf{Bộ kiểm thử}}	& \textbf{Mô hình}	& \textbf{STOI}	& \textbf{NBPESQ}	& \textbf{WBPESQ}	& \textbf{SIG}	& \textbf{BAK} \\
				\hline
				\multirow{4}{*}{$\name{SNR}=-5$}        &Baseline       & 0.64 (0.11)   & 1.36 (0.27)   & 1.10 (0.12)   & 3.66 (0.34)   & 0.98 (0.71) \\
				&FullSubNet     & 0.54 (0.09)   & 1.22 (0.12)   & 1.06 (0.04)   & 2.85 (0.34)   & 1.87 (0.72) \\
				&DCCRN  & 0.46 (0.09)   & 1.19 (0.15)   & 1.07 (0.06)   & 2.43 (0.33)   & 2.18 (0.58) \\
				&DTLN   & 0.70 (0.12)   & 1.64 (0.36)   & 1.29 (0.21)   & 3.35 (0.37)   & 2.96 (0.43) \\
				&&&&&&\\
				\multirow{4}{*}{$\name{SNR}=0$}         &Baseline       & 0.75 (0.09)   & 1.54 (0.36)   & 1.16 (0.19)   & 3.89 (0.27)   & 1.21 (0.73) \\
				&FullSubNet     & 0.60 (0.07)   & 1.28 (0.13)   & 1.08 (0.05)   & 2.78 (0.36)   & 2.29 (0.57) \\
				&DCCRN  & 0.55 (0.09)   & 1.31 (0.21)   & 1.11 (0.09)   & 2.66 (0.45)   & 2.37 (0.56) \\
				&DTLN   & 0.82 (0.09)   & 2.07 (0.40)   & 1.55 (0.29)   & 3.59 (0.32)   & 3.29 (0.38) \\
				&&&&&&\\
				\multirow{4}{*}{$\name{SNR}=5$}         &Baseline       & 0.84 (0.08)   & 1.78 (0.41)   & 1.29 (0.28)   & 4.15 (0.20)   & 1.63 (0.76) \\
				&FullSubNet     & 0.64 (0.06)   & 1.37 (0.14)   & 1.11 (0.07)   & 2.70 (0.36)   & 2.65 (0.46) \\
				&DCCRN  & 0.64 (0.08)   & 1.46 (0.25)   & 1.17 (0.13)   & 2.97 (0.53)   & 2.71 (0.50) \\
				&DTLN   & 0.89 (0.07)   & 2.51 (0.39)   & 1.85 (0.34)   & 3.75 (0.32)   & 3.61 (0.33) \\
				&&&&&&\\
				\multirow{4}{*}{$\name{SNR}=10$}        &Baseline       & 0.90 (0.07)   & 2.11 (0.46)   & 1.54 (0.37)   & 4.32 (0.15)   & 2.20 (0.69) \\
				&FullSubNet     & 0.66 (0.06)   & 1.47 (0.16)   & 1.16 (0.08)   & 2.67 (0.37)   & 2.99 (0.39) \\
				&DCCRN  & 0.69 (0.08)   & 1.64 (0.30)   & 1.25 (0.16)   & 3.20 (0.54)   & 3.13 (0.46) \\
				&DTLN   & 0.93 (0.07)   & 2.87 (0.39)   & 2.15 (0.40)   & 3.90 (0.32)   & 3.85 (0.29) \\
				&&&&&&\\
				\multirow{4}{*}{$\name{SNR}=15$}        &Baseline       & 0.94 (0.05)   & 2.57 (0.50)   & 1.97 (0.48)   & 4.39 (0.15)   & 2.75 (0.60) \\
				&FullSubNet     & 0.68 (0.07)   & 1.55 (0.19)   & 1.20 (0.11)   & 2.69 (0.39)   & 3.25 (0.36) \\
				&DCCRN  & 0.73 (0.09)   & 1.86 (0.39)   & 1.36 (0.23)   & 3.34 (0.55)   & 3.44 (0.44) \\
				&DTLN   & 0.95 (0.05)   & 3.26 (0.36)   & 2.52 (0.43)   & 3.99 (0.32)   & 4.01 (0.25) \\
				&&&&&&\\
				\multirow{4}{*}{$\name{SNR}=20$}        &Baseline       & 0.96 (0.04)   & 3.08 (0.48)   & 2.51 (0.51)   & 4.38 (0.19)   & 3.25 (0.46) \\
				&FullSubNet     & 0.69 (0.07)   & 1.61 (0.23)   & 1.25 (0.15)   & 2.73 (0.44)   & 3.45 (0.35) \\
				&DCCRN  & 0.75 (0.09)   & 2.07 (0.47)   & 1.47 (0.30)   & 3.38 (0.55)   & 3.65 (0.42) \\
				&DTLN   & 0.96 (0.05)   & 3.56 (0.35)   & 2.86 (0.47)   & 4.07 (0.32)   & 4.13 (0.24) \\
				\hline
			\end{tabular}
			\caption{So sánh các mô hình thông qua các metrics trên bộ kiểm thử tiếng Việt}
			\label{re::competitors_metrics}
		\end{table}
	
	Sau các phiên bản cải tiến khác nhau, mô hình Post của chúng tôi cũng đã đạt được một số kết quả đáng ghi nhận. Thông qua các số liệu có trong \tableref{re::compare_models}, chúng tôi có thể đánh giá được mô hình của chúng tôi tuy với lượng tham số chỉ bằng 1/9 lượng tham số của các mô hình đã có (khoảng gần 1 triệu tham số) và lượng dữ liệu dùng để huấn luyện mô hình chỉ bằng 1/5 so với của họ (mô hình của chúng tôi được huấn luyện trên khoảng 100 giờ dữ liệu đã lai tạo, trong khi mô hình đã có được huấn luyện trên bộ dữ liệu trong cuộc thi DNS gồm khoảng 500 giờ dữ liệu sạch) nhưng mô hình chúng tôi vẫn đạt được mức độ đánh giá sự nghe hiểu (STOI) tương đương với đối thủ (chênh lệch của STOI giữa mô hình chúng tôi và của đối thủ ít hơn 2\%). Dù chất lượng âm thanh được đánh giá qua PESQ (cả wideband PESQ và narrowband PESQ) vẫn chưa thể vượt qua nhưng mức độ lọc nhiễu (BAK) từ âm thanh chúng tôi vẫn đạt được mức cao hơn hẳn mô hình đối thủ và hạn chế được sự mất mát của chất lượng giọng nói (SIG).
	
		\begin{table}[h]
			\centering
			\begin{tabular}{c c c c c c}
				\hline
				\textbf{Trường hợp}	& \textbf{Số mẫu}	& \textbf{Độ trễ}	& \textbf{Tứ phân vị 25\%}	& \textbf{Tứ phân vị 50\%}	& \textbf{Tứ phân vị 75\%} \\
				\hline
				(1.1)				& 34179				& 1.76 (0.76) ms	& 1.54 ms					& 1.62 ms			& 1.75 ms \\
				(1.2)				& 35787				& 1.74 (0.48) ms	& 1.56 ms					& 1.65 ms			& 1.78 ms \\
				(1.3)				& 42330				& 1.66 (0.37) ms	& 1.53 ms					& 1.61 ms			& 1.72 ms \\
				(2)					& 31578				& 1.66 (0.44) ms	& 1.51 ms					& 1.60 ms			& 1.69 ms \\
				(3)					& 36076				& 1.62 (0.44) ms	& 1.48 ms					& 1.56 ms			& 1.66 ms \\
				\hline
			\end{tabular}
			\caption{Độ trễ của pipeline lọc nhiễu động trong các trường hợp kiểm thử}
			\label{re::svc_latency_detail}
		\end{table}
	
		\begin{table}[h]
			\centering
			\begin{tabular}{l c c c c c c}
				\hline
				\multicolumn{1}{c}{\textbf{Bộ kiểm thử}}	& \textbf{Mô hình}	& \textbf{STOI}	& \textbf{NBPESQ}	& \textbf{WBPESQ}	& \textbf{SIG}	& \textbf{BAK} \\
				\hline
				\multirow{8}{*}{$\name{SNR}=-5$}        &Baseline       & 0.64 (0.11)   & 1.36 (0.27)   & 1.10 (0.12)   & 3.66 (0.34)   & 0.98 (0.71) \\
				&DTLN   & 0.70 (0.12)   & 1.64 (0.36)   & 1.29 (0.21)   & 3.35 (0.37)   & 2.96 (0.43) \\
				&DTLN reduced   & 0.52 (0.16)   & 1.22 (0.21)   & 1.14 (0.15)   & 2.54 (0.41)   & 3.15 (0.28) \\
				&DTLN Post      & 0.49 (0.18)   & 1.17 (0.16)   & 1.11 (0.09)   & 2.38 (0.41)   & 3.39 (0.28) \\
				&Post v1        & 0.59 (0.13)   & 1.25 (0.22)   & 1.13 (0.12)   & 2.52 (0.46)   & 2.97 (0.38) \\
				&Post v2        & 0.64 (0.13)   & 1.38 (0.26)   & 1.19 (0.16)   & 2.83 (0.43)   & 3.09 (0.42) \\
				&Post v3 static & 0.68 (0.12)   & 1.45 (0.29)   & 1.22 (0.17)   & 2.99 (0.42)   & \textbf{3.42 (0.38)} \\
				&Post v3 dynamic        & 0.68 (0.12)   & 1.41 (0.27)   & 1.19 (0.15)   & 3.02 (0.43)   & \textbf{3.52 (0.38)} \\
				&&&&&&\\
				\multirow{8}{*}{$\name{SNR}=0$}         &Baseline       & 0.75 (0.09)   & 1.54 (0.36)   & 1.16 (0.19)   & 3.89 (0.27)   & 1.21 (0.73) \\
				&DTLN   & 0.82 (0.09)   & 2.07 (0.40)   & 1.55 (0.29)   & 3.59 (0.32)   & 3.29 (0.38) \\
				&DTLN reduced   & 0.69 (0.15)   & 1.49 (0.33)   & 1.30 (0.23)   & 3.06 (0.42)   & 3.34 (0.33) \\
				&DTLN Post      & 0.68 (0.16)   & 1.39 (0.30)   & 1.23 (0.18)   & 2.88 (0.44)   & 3.54 (0.33) \\
				&Post v1        & 0.77 (0.10)   & 1.60 (0.38)   & 1.32 (0.24)   & 3.03 (0.42)   & 3.23 (0.42) \\
				&Post v2        & 0.79 (0.10)   & 1.74 (0.37)   & 1.40 (0.27)   & 3.27 (0.38)   & 3.39 (0.43) \\
				&Post v3 static & 0.81 (0.08)   & 1.77 (0.37)   & 1.41 (0.27)   & 3.35 (0.36)   & \textbf{3.77 (0.33)} \\
				&Post v3 dynamic        & 0.81 (0.08)   & 1.74 (0.35)   & 1.37 (0.24)   & 3.42 (0.36)   & \textbf{3.83 (0.33)} \\
				&&&&&&\\
				\multirow{8}{*}{$\name{SNR}=5$}         &Baseline       & 0.84 (0.08)   & 1.78 (0.41)   & 1.29 (0.28)   & 4.15 (0.20)   & 1.63 (0.76) \\
				&DTLN   & 0.89 (0.07)   & 2.51 (0.39)   & 1.85 (0.34)   & 3.75 (0.32)   & 3.61 (0.33) \\
				&DTLN reduced   & 0.80 (0.13)   & 1.83 (0.44)   & 1.52 (0.34)   & 3.39 (0.37)   & 3.58 (0.33) \\
				&DTLN Post      & 0.80 (0.13)   & 1.69 (0.39)   & 1.40 (0.26)   & 3.24 (0.40)   & 3.72 (0.32) \\
				&Post v1        & 0.86 (0.07)   & 1.95 (0.35)   & 1.52 (0.26)   & 3.32 (0.36)   & 3.46 (0.37) \\
				&Post v2        & 0.87 (0.07)   & 2.08 (0.36)   & 1.61 (0.29)   & 3.52 (0.37)   & 3.69 (0.34) \\
				&Post v3 static & \textbf{0.89 (0.07)}   & 2.14 (0.38)   & 1.65 (0.32)   & 3.65 (0.34)   & \textbf{4.01 (0.26)} \\
				&Post v3 dynamic        & \textbf{0.89 (0.07)}   & 2.10 (0.37)   & 1.59 (0.29)   & 3.70 (0.35)   & \textbf{4.06 (0.27)} \\
				&&&&&&\\
				\multirow{8}{*}{$\name{SNR}=10$}        &Baseline       & 0.90 (0.07)   & 2.11 (0.46)   & 1.54 (0.37)   & 4.32 (0.15)   & 2.20 (0.69) \\
				&DTLN   & 0.93 (0.07)   & 2.87 (0.39)   & 2.15 (0.40)   & 3.90 (0.32)   & 3.85 (0.29) \\
				&DTLN reduced   & 0.87 (0.11)   & 2.20 (0.51)   & 1.79 (0.42)   & 3.59 (0.34)   & 3.79 (0.32) \\
				&DTLN Post      & 0.86 (0.11)   & 2.02 (0.46)   & 1.62 (0.34)   & 3.45 (0.36)   & 3.91 (0.30) \\
				&Post v1        & 0.91 (0.07)   & 2.33 (0.42)   & 1.80 (0.34)   & 3.52 (0.36)   & 3.71 (0.34) \\
				&Post v2        & 0.92 (0.06)   & 2.44 (0.40)   & 1.90 (0.35)   & 3.74 (0.36)   & 3.92 (0.32) \\
				&Post v3 static & 0.92 (0.06)   & 2.49 (0.42)   & 1.90 (0.39)   & 3.89 (0.32)   & \textbf{4.16 (0.24)} \\
				&Post v3 dynamic        & 0.92 (0.06)   & 2.45 (0.40)   & 1.83 (0.36)   & \textbf{3.91 (0.33)}   & \textbf{4.20 (0.24)} \\
				&&&&&&\\
				\multirow{8}{*}{$\name{SNR}=15$}        &Baseline       & 0.94 (0.05)   & 2.57 (0.50)   & 1.97 (0.48)   & 4.39 (0.15)   & 2.75 (0.60) \\
				&DTLN   & 0.95 (0.05)   & 3.26 (0.36)   & 2.52 (0.43)   & 3.99 (0.32)   & 4.01 (0.25) \\
				&DTLN reduced   & 0.89 (0.10)   & 2.48 (0.56)   & 1.99 (0.51)   & 3.67 (0.31)   & 3.93 (0.26) \\
				&DTLN Post      & 0.89 (0.10)   & 2.28 (0.52)   & 1.79 (0.43)   & 3.53 (0.33)   & 4.00 (0.25) \\
				&Post v1        & 0.93 (0.06)   & 2.67 (0.46)   & 2.05 (0.41)   & 3.66 (0.36)   & 3.87 (0.31) \\
				&Post v2        & 0.93 (0.06)   & 2.76 (0.46)   & 2.14 (0.43)   & 3.88 (0.36)   & 4.08 (0.27) \\
				&Post v3 static & \textbf{0.95 (0.05)}   & 2.90 (0.44)   & 2.23 (0.45)   & \textbf{4.04 (0.30)}   & \textbf{4.26 (0.22)} \\
				&Post v3 dynamic        & \textbf{0.95 (0.05)}   & 2.83 (0.41)   & 2.13 (0.41)   & \textbf{4.04 (0.32)}   & \textbf{4.29 (0.21)} \\
				&&&&&&\\
				\multirow{8}{*}{$\name{SNR}=20$}        &Baseline       & 0.96 (0.04)   & 3.08 (0.48)   & 2.51 (0.51)   & 4.38 (0.19)   & 3.25 (0.46) \\
				&DTLN   & 0.96 (0.05)   & 3.56 (0.35)   & 2.86 (0.47)   & 4.07 (0.32)   & 4.13 (0.24) \\
				&DTLN reduced   & 0.91 (0.09)   & 2.73 (0.60)   & 2.18 (0.57)   & 3.76 (0.31)   & 4.02 (0.26) \\
				&DTLN Post      & 0.91 (0.09)   & 2.51 (0.57)   & 1.97 (0.49)   & 3.63 (0.32)   & 4.07 (0.26) \\
				&Post v1        & 0.94 (0.05)   & 2.96 (0.49)   & 2.31 (0.47)   & 3.76 (0.35)   & 4.00 (0.31) \\
				&Post v2        & 0.95 (0.05)   & 3.07 (0.47)   & 2.44 (0.47)   & 3.99 (0.35)   & 4.19 (0.25) \\
				&Post v3 static & \textbf{0.96 (0.05)}   & 3.22 (0.47)   & 2.49 (0.51)   & \textbf{4.14 (0.30)}   & \textbf{4.33 (0.21)} \\
				&Post v3 dynamic        & \textbf{0.96 (0.05)}   & 3.12 (0.44)   & 2.38 (0.48)   & \textbf{4.14 (0.30)}   & \textbf{4.35 (0.20)} \\
				\hline
			\end{tabular}
			\caption{So sánh các metrics giữa các mô hình trên bộ kiểm thử tiếng Việt}
			\label{re::compare_models}
		\end{table}
	
	% multilingual
	
	Để có thể so sánh một cách toàn diện mô hình được chúng tôi đề xuất với nhiều loại ngôn ngữ ở các ngưỡng \name{SNR} khác nhau, chúng tôi đã thực hiện kiểm thử với bộ dữ liệu đa ngôn ngữ được chúng tôi chuẩn bị ở phần trước. Ngoài ra cũng để có một cái nhìn khách quan hơn về chất lượng của mô hình chúng tôi với các sản phẩn mã nguồn mở và cả các sản phẩm thương mại, thông qua quá trình khảo sát chúng tôi chọn ra một số ứng dụng lọc nhiễu để so sánh với kết quả của mình bao gồm Sox\footnote{\url{http://sox.sourceforge.net/sox.html}} và CrystalSound\footnote{\url{https://crystalsound.ai/}}. Kết quả của việc kiểm thử mô hình của chúng tôi trên bộ dữ liệu đa ngôn ngữ và so sánh trên bộ dữ liệu tiếng Việt với Sox và CrystalSound được chúng tôi trình bày ở \tableref{re::compare_models_multilingual} và \tableref{re::compare_models_prebuilt}.
	
		\begin{table}[h]
			\centering
			\begin{tabular}{l c c c c c c}
				\hline
				\multicolumn{1}{c}{\textbf{Bộ kiểm thử}}	& \textbf{Mô hình}	& \textbf{STOI}	& \textbf{NBPESQ}	& \textbf{WBPESQ}	& \textbf{SIG}	& \textbf{BAK} \\
				\hline
				\multirow{6}{*}{$\name{SNR}=-5$}        &Baseline       & 0.64 (0.11)   & 1.36 (0.27)   & 1.10 (0.12)   & 3.66 (0.34)  & 0.98 (0.71) \\
				&Sox    & 0.48 (0.08)   & 1.16 (0.12)   & 1.06 (0.06)   & 3.04 (0.31)   & 2.90 (0.56) \\
				&\multirow{2}{*}{CrystalSound}   & 0.44 (0.07)   & 1.78 (0.35)   & 1.40 (0.25)   & 3.49 (0.35)   & 3.03 (0.49) \\
				&       &  -  & 1.82 (0.36)   & 1.44 (0.25)   & 3.38 (0.34)   & 3.00 (0.44) \\
				&Post v3 static & \textbf{0.68 (0.12)}   & 1.45 (0.29)   & 1.22 (0.17)   & 2.99 (0.42)   & \textbf{3.42 (0.38)} \\
				&Post v3 dynamic        & \textbf{0.68 (0.12)}   & 1.41 (0.27)   & 1.19 (0.15)   & 3.02 (0.43)   & \textbf{3.52 (0.38)} \\
				&&&&&&\\
				\multirow{6}{*}{$\name{SNR}=0$}         &Baseline       & 0.75 (0.09)   & 1.54 (0.36)   & 1.16 (0.19)   & 3.89 (0.27)  & 1.21 (0.73) \\
				&Sox    & 0.58 (0.08)   & 1.23 (0.13)   & 1.09 (0.06)   & 3.11 (0.27)   & 3.25 (0.47) \\
				&\multirow{2}{*}{CrystalSound}   & 0.48 (0.06)   & 2.18 (0.41)   & 1.70 (0.33)   & 3.68 (0.30)   & 3.37 (0.42) \\
				&       &  -  & 2.23 (0.44)   & 1.75 (0.35)   & 3.55 (0.30)   & 3.32 (0.40) \\
				&Post v3 static & \textbf{0.81 (0.08)}   & 1.77 (0.37)   & 1.41 (0.27)   & 3.35 (0.36)   & \textbf{3.77 (0.33)} \\
				&Post v3 dynamic        & \textbf{0.81 (0.08)}   & 1.74 (0.35)   & 1.37 (0.24)   & 3.42 (0.36)   & \textbf{3.83 (0.33)} \\
				&&&&&&\\
				\multirow{6}{*}{$\name{SNR}=5$}         &Baseline       & 0.84 (0.08)   & 1.78 (0.41)   & 1.29 (0.28)   & 4.15 (0.20)  & 1.63 (0.76) \\
				&Sox    & 0.63 (0.07)   & 1.32 (0.17)   & 1.13 (0.09)   & 3.15 (0.25)   & 3.58 (0.34) \\
				&\multirow{2}{*}{CrystalSound}   & 0.50 (0.06)   & 2.51 (0.37)   & 1.97 (0.34)   & 3.82 (0.31)   & 3.62 (0.36) \\
				&       &  -  & 2.55 (0.41)   & 2.02 (0.38)   & 3.72 (0.31)   & 3.57 (0.35) \\
				&Post v3 static & \textbf{0.89 (0.07)}   & 2.14 (0.38)   & 1.65 (0.32)   & 3.65 (0.34)   & \textbf{4.01 (0.26)} \\
				&Post v3 dynamic        & \textbf{0.89 (0.07)}   & 2.10 (0.37)   & 1.59 (0.29)   & 3.70 (0.35)   & \textbf{4.06 (0.27)} \\
				&&&&&&\\
				\multirow{6}{*}{$\name{SNR}=10$}        &Baseline       & 0.90 (0.07)   & 2.11 (0.46)   & 1.54 (0.37)   & 4.32 (0.15)  & 2.20 (0.69) \\
				&Sox    & 0.65 (0.08)   & 1.36 (0.19)   & 1.16 (0.10)   & 3.13 (0.27)   & 3.78 (0.32) \\
				&\multirow{2}{*}{CrystalSound}   & 0.51 (0.06)   & 2.81 (0.40)   & 2.23 (0.38)   & 3.93 (0.30)   & 3.81 (0.34) \\
				&       & -  & 2.87 (0.43)   & 2.31 (0.42)   & 3.87 (0.30)   & 3.80 (0.36) \\
				&Post v3 static & \textbf{0.92 (0.06)}   & 2.49 (0.42)   & 1.90 (0.39)   & 3.89 (0.32)   & \textbf{4.16 (0.24)} \\
				&Post v3 dynamic        & \textbf{0.92 (0.06)}   & 2.45 (0.40)   & 1.83 (0.36)   & 3.91 (0.33)   & \textbf{4.20 (0.24)} \\
				&&&&&&\\
				\multirow{6}{*}{$\name{SNR}=15$}        &Baseline       & 0.94 (0.05)   & 2.57 (0.50)   & 1.97 (0.48)   & 4.39 (0.15)  & 2.75 (0.60) \\
				&Sox    & 0.65 (0.08)   & 1.39 (0.21)   & 1.20 (0.13)   & 3.11 (0.29)   & 3.85 (0.31) \\
				&\multirow{2}{*}{CrystalSound}   & 0.51 (0.05)   & 3.07 (0.48)   & 2.46 (0.42)   & 4.00 (0.28)   & 3.89 (0.30) \\
				&       & -  & 3.18 (0.45)   & 2.59 (0.46)   & 3.95 (0.29)   & 3.90 (0.30) \\
				&Post v3 static & \textbf{0.95 (0.05)}   & 2.90 (0.44)   & 2.23 (0.45)   & \textbf{4.04 (0.30)}   & \textbf{4.26 (0.22)} \\
				&Post v3 dynamic        & \textbf{0.95 (0.05)}   & 2.83 (0.41)   & 2.13 (0.41)   & \textbf{4.04 (0.32)}   & \textbf{4.29 (0.21)} \\
				&&&&&&\\
				\multirow{6}{*}{$\name{SNR}=20$}        &Baseline       & 0.96 (0.04)   & 3.08 (0.48)   & 2.51 (0.51)   & 4.38 (0.19)  & 3.25 (0.46) \\
				&Sox    & 0.66 (0.08)   & 1.39 (0.21)   & 1.20 (0.11)   & 3.10 (0.26)   & 3.92 (0.31) \\
				&\multirow{2}{*}{CrystalSound}   & 0.51 (0.06)   & 3.30 (0.48)   & 2.64 (0.46)   & 4.01 (0.29)   & 3.94 (0.30) \\
				&       &  -  & 3.47 (0.44)   & 2.89 (0.48)   & 4.00 (0.29)   & 3.97 (0.31) \\
				&Post v3 static & \textbf{0.96 (0.05)}   & 3.22 (0.47)   & 2.49 (0.51)   & \textbf{4.14 (0.30)}   & \textbf{4.33 (0.21)} \\
				&Post v3 dynamic        & \textbf{0.96 (0.05)}   & 3.12 (0.44)   & 2.38 (0.48)   & \textbf{4.14 (0.30)}   & \textbf{4.35 (0.20)} \\
				\hline
			\end{tabular}
			\caption{So sánh các metrics giữa mô hình đề xuất và các ứng dụng Sox phiên bản 14.4.2, CrystalSound phiên bản 0.15.2 (dòng trên) và phiên bản 1.4.0.0 (dòng dưới) trên bộ kiểm thử tiếng Việt}
			\label{re::compare_models_prebuilt}
		\end{table}
	
	Ngoài vấn đề về chất lượng, độ trễ của mô hình động và pipeline chạy lọc nhiễu cũng rất quan trọng do vậy chúng tôi đã tiến hành đo đạc một số kết quả độ trễ của toàn bộ pipeline trong các trường hợp: \textit{Người dùng đang nói chuyện thông qua Google Meet} (1), \textit{Người dùng đang ghi âm bằng ứng dụng của chúng tôi} (2) và \textit{Người dùng đang sử dụng ứng dụng nào cả} (3). Trong đó với trường hợp \textit{Người dùng đang nói chuyện thông qua Google Meet} chúng tôi lại chia thêm thành các trường hợp con như \textit{Người dùng chia sẻ màn hình (đang mở cửa sổ chia sẻ)} (1.1), \textit{Người dùng chia sẻ màn hình (đang ẩn cửa sổ chia sẻ)} (1.2) và \textit{Người dùng chỉ đang trò chuyện} (1.3). Với từng trường hợp như vậy chúng tôi thực nghiệm chạy thử trên máy ảo có CPU 4 cores, RAM 4GB và chạy trên hệ điều hành Windows 10, chúng tôi thu được độ trễ của pipeline được thể hiện như trong \tableref{re::svc_latency_detail}.
	
		\begin{table}[h]
			\centering
			\begin{tabular}{l l c c c c c}
				\hline
				\multicolumn{1}{c}{\textbf{Bộ kiểm thử}}	& \multicolumn{1}{c}{\textbf{Ngôn ngữ}}	& \textbf{STOI}	& \textbf{NBPESQ}	& \textbf{WBPESQ}	& \textbf{SIG}	& \textbf{BAK} \\
				\hline
				\multirow{6}{*}{$\name{SNR}=-5$}        & Japanese       & 0.53 (0.09)   & 1.39 (0.41)   & 1.19 (0.37)   & 3.51 (0.31)   & 1.01 (0.67) \\
				& French       & 0.58 (0.12)   & 1.32 (0.27)   & 1.10 (0.17)   & 3.62 (0.33)   & 1.15 (0.71) \\
				& Russian        & 0.60 (0.11)   & 1.29 (0.27)   & 1.08 (0.14)   & 3.60 (0.30)   & 1.21 (0.72) \\
				& Germany        & 0.63 (0.10)   & 1.36 (0.23)   & 1.11 (0.19)   & 3.63 (0.32)   & 1.11 (0.67) \\
				& Portuguese       & 0.57 (0.10)   & 1.40 (0.22)   & 1.09 (0.10)   & 3.62 (0.28)   & 1.15 (0.63) \\
				& Chinese        & 0.52 (0.11)   & 1.38 (0.33)   & 1.10 (0.21)   & 3.64 (0.31)   & 1.07 (0.66) \\
				&&&&&&\\
				\multirow{6}{*}{$\name{SNR}=0$}         & Japanese       & 0.63 (0.09)   & 1.38 (0.32)   & 1.13 (0.19)   & 3.63 (0.30)   & 1.09 (0.60) \\
				& French       & 0.68 (0.12)   & 1.40 (0.29)   & 1.10 (0.12)   & 3.81 (0.28)   & 1.34 (0.71) \\
				& Russian        & 0.69 (0.10)   & 1.34 (0.24)   & 1.08 (0.08)   & 3.78 (0.27)   & 1.33 (0.67) \\
				& Germany        & 0.74 (0.09)   & 1.46 (0.27)   & 1.12 (0.14)   & 3.83 (0.29)   & 1.30 (0.68) \\
				& Portuguese       & 0.68 (0.10)   & 1.49 (0.36)   & 1.13 (0.22)   & 3.80 (0.25)   & 1.36 (0.71) \\
				& Chinese        & 0.62 (0.12)   & 1.51 (0.36)   & 1.11 (0.16)   & 3.82 (0.26)   & 1.28 (0.68) \\
				&&&&&&\\
				\multirow{6}{*}{$\name{SNR}=5$}         & Japanese       & 0.73 (0.08)   & 1.45 (0.32)   & 1.15 (0.17)   & 3.80 (0.27)   & 1.33 (0.67) \\
				& French       & 0.78 (0.12)   & 1.56 (0.36)   & 1.18 (0.23)   & 4.05 (0.24)   & 1.70 (0.73) \\
				& Russian        & 0.77 (0.10)   & 1.53 (0.36)   & 1.15 (0.21)   & 3.96 (0.30)   & 1.79 (0.77) \\
				& Germany        & 0.83 (0.07)   & 1.65 (0.35)   & 1.22 (0.24)   & 4.07 (0.23)   & 1.69 (0.78) \\
				& Portuguese       & 0.77 (0.08)   & 1.70 (0.36)   & 1.22 (0.22)   & 4.02 (0.23)   & 1.68 (0.70) \\
				& Chinese        & 0.70 (0.12)   & 1.73 (0.40)   & 1.21 (0.20)   & 4.03 (0.22)   & 1.57 (0.68) \\
				&&&&&&\\
				\multirow{6}{*}{$\name{SNR}=10$}        & Japanese       & 0.83 (0.07)   & 1.66 (0.42)   & 1.27 (0.31)   & 4.00 (0.21)   & 1.66 (0.69) \\
				& French       & 0.86 (0.10)   & 1.89 (0.47)   & 1.37 (0.30)   & 4.23 (0.20)   & 2.08 (0.72) \\
				& Russian        & 0.85 (0.09)   & 1.76 (0.45)   & 1.29 (0.29)   & 4.12 (0.31)   & 2.18 (0.75) \\
				& Germany        & 0.90 (0.05)   & 1.96 (0.38)   & 1.40 (0.27)   & 4.23 (0.19)   & 2.12 (0.71) \\
				& Portuguese       & 0.85 (0.08)   & 2.01 (0.46)   & 1.43 (0.34)   & 4.17 (0.21)   & 2.14 (0.70) \\
				& Chinese        & 0.78 (0.13)   & 2.12 (0.53)   & 1.47 (0.41)   & 4.17 (0.19)   & 1.97 (0.79) \\
				&&&&&&\\
				\multirow{6}{*}{$\name{SNR}=15$}        & Japanese       & 0.90 (0.05)   & 1.96 (0.48)   & 1.48 (0.41)   & 4.18 (0.15)   & 2.12 (0.69) \\
				& French       & 0.91 (0.08)   & 2.27 (0.51)   & 1.69 (0.41)   & 4.33 (0.18)   & 2.59 (0.65) \\
				& Russian        & 0.90 (0.07)   & 2.02 (0.51)   & 1.50 (0.39)   & 4.17 (0.36)   & 2.57 (0.65) \\
				& Germany        & 0.94 (0.04)   & 2.37 (0.45)   & 1.77 (0.42)   & 4.33 (0.17)   & 2.68 (0.65) \\
				& Portuguese       & 0.90 (0.07)   & 2.45 (0.55)   & 1.82 (0.49)   & 4.24 (0.21)   & 2.57 (0.65) \\
				& Chinese        & 0.83 (0.13)   & 2.53 (0.51)   & 1.84 (0.42)   & 4.23 (0.17)   & 2.41 (0.69) \\
				&&&&&&\\
				\multirow{6}{*}{$\name{SNR}=20$}        & Japanese       & 0.95 (0.03)   & 2.36 (0.55)   & 1.85 (0.55)   & 4.26 (0.14)   & 2.68 (0.57) \\
				& French       & 0.95 (0.05)   & 2.68 (0.51)   & 2.13 (0.48)   & 4.35 (0.19)   & 3.09 (0.54) \\
				& Russian        & 0.94 (0.07)   & 2.55 (0.59)   & 2.01 (0.58)   & 4.26 (0.31)   & 3.13 (0.50) \\
				& Germany        & 0.97 (0.03)   & 2.76 (0.45)   & 2.21 (0.46)   & 4.35 (0.19)   & 3.11 (0.57) \\
				& Portuguese       & 0.94 (0.05)   & 2.81 (0.54)   & 2.25 (0.53)   & 4.26 (0.20)   & 3.02 (0.52) \\
				& Chinese        & 0.89 (0.10)   & 3.00 (0.49)   & 2.38 (0.47)   & 4.25 (0.15)   & 2.84 (0.62) \\
				\hline
			\end{tabular}
			\caption{Metric trên từng bộ kiểm thử đa ngôn ngữ}
			\label{re::multilingual_baseline}
		\end{table}
	
		\begin{sidewaystable}
			\centering
			\begin{subtable}{\textwidth}
				\centering
				\begin{tabular}{llcccccc}
					\hline
					\multicolumn{1}{c}{\textbf{Bộ kiểm thử}}	& \multicolumn{1}{c}{\textbf{Ngôn ngữ}}	& \textbf{Mô hình}	& \textbf{STOI}	& \textbf{NBPESQ}	& \textbf{WBPESQ}	& \textbf{SIG}	& \textbf{BAK} \\
					\hline
					\multirow{27}{*}{$\name{SNR}=-5$}       & \multirow{3}{*}{Vietnamese}     &Baseline       & 0.64 (0.11)   & 1.36 (0.27)   & 1.10 (0.12)   & 3.66 (0.34)   & 0.98 (0.71) \\
					&               &Post v3 static & 0.68 (0.12)   & 1.45 (0.29)   & 1.22 (0.17)   & 2.99 (0.42)   & 3.42 (0.38) \\
					&               &Post v3 dynamic        & 0.68 (0.12)   & 1.41 (0.27)   & 1.19 (0.15)   & 3.02 (0.43)   & 3.52 (0.38) \\
					&&&&&&&\\
					& \multirow{3}{*}{Japanese}     &Baseline       & 0.53 (0.09)   & 1.39 (0.41)   & 1.19 (0.37)   & 3.51 (0.31)   & 1.01 (0.67) \\
					&               &Post v3 static & 0.54 (0.10)   & 1.31 (0.20)   & 1.12 (0.09)   & 2.87 (0.40)   & 3.35 (0.46) \\
					&               &Post v3 dynamic        & 0.54 (0.10)   & 1.33 (0.20)   & 1.14 (0.10)   & 2.83 (0.34)   & 3.62 (0.33) \\
					&&&&&&&\\
					& \multirow{3}{*}{French}       &Baseline       & 0.58 (0.12)   & 1.32 (0.27)   & 1.10 (0.17)   & 3.62 (0.33)   & 1.15 (0.71) \\
					&               &Post v3 static & 0.61 (0.13)   & 1.37 (0.22)   & 1.13 (0.09)   & 2.99 (0.44)   & 3.48 (0.39) \\
					&               &Post v3 dynamic        & 0.62 (0.13)   & 1.37 (0.22)   & 1.14 (0.10)   & 2.93 (0.39)   & 3.61 (0.33) \\
					&&&&&&&\\
					& \multirow{3}{*}{Russian}      &Baseline       & 0.60 (0.11)   & 1.29 (0.27)   & 1.08 (0.14)   & 3.60 (0.30)   & 1.21 (0.72) \\
					&               &Post v3 static & 0.60 (0.12)   & 1.35 (0.23)   & 1.11 (0.12)   & 2.91 (0.46)   & 3.46 (0.36) \\
					&               &Post v3 dynamic        & 0.60 (0.13)   & 1.34 (0.23)   & 1.12 (0.11)   & 2.84 (0.44)   & 3.57 (0.35) \\
					&&&&&&&\\
					& \multirow{3}{*}{Germany}      &Baseline       & 0.63 (0.10)   & 1.36 (0.23)   & 1.11 (0.19)   & 3.63 (0.32)   & 1.11 (0.67) \\
					&               &Post v3 static & 0.67 (0.11)   & 1.36 (0.23)   & 1.13 (0.12)   & 3.05 (0.42)   & 3.51 (0.39) \\
					&               &Post v3 dynamic        & 0.67 (0.11)   & 1.35 (0.22)   & 1.13 (0.11)   & 3.00 (0.37)   & 3.67 (0.32) \\
					&&&&&&&\\
					& \multirow{3}{*}{Portuguese}   &Baseline       & 0.57 (0.10)   & 1.40 (0.22)   & 1.09 (0.10)   & 3.62 (0.28)   & 1.15 (0.63) \\
					&               &Post v3 static & 0.62 (0.10)   & 1.39 (0.20)   & 1.12 (0.08)   & 3.02 (0.41)   & 3.51 (0.39) \\
					&               &Post v3 dynamic        & 0.62 (0.11)   & 1.36 (0.19)   & 1.12 (0.08)   & 2.96 (0.38)   & 3.68 (0.31) \\
					&&&&&&&\\
					& \multirow{3}{*}{Chinese}      &Baseline       & 0.52 (0.11)   & 1.38 (0.33)   & 1.10 (0.21)   & 3.64 (0.31)   & 1.07 (0.66) \\
					&               &Post v3 static & 0.53 (0.12)   & 1.40 (0.28)   & 1.14 (0.13)   & 2.97 (0.48)   & 3.37 (0.40) \\
					&               &Post v3 dynamic        & 0.54 (0.12)   & 1.39 (0.26)   & 1.14 (0.12)   & 2.89 (0.41)   & 3.47 (0.36) \\
					\hline
				\end{tabular}
				\caption{So sánh các metrics của mô hình đề xuất với các ngôn ngữ khác nhau có $\name{SNR}=-5$}
			\end{subtable}
		
		\caption{Metrics mô hình đề xuất với các loại ngôn ngữ}
		\label{re::compare_models_multilingual}
		\end{sidewaystable}
	
		\clearpage
	
		\begin{sidewaystable} \ContinuedFloat
			\centering
			\begin{subtable}{\textwidth}
				\centering
				\begin{tabular}{llcccccc}
					\hline
					\multicolumn{1}{c}{\textbf{Bộ kiểm thử}}	& \multicolumn{1}{c}{\textbf{Ngôn ngữ}}	& \textbf{Mô hình}	& \textbf{STOI}	& \textbf{NBPESQ}	& \textbf{WBPESQ}	& \textbf{SIG}	& \textbf{BAK} \\
					\hline
					\multirow{27}{*}{$\name{SNR}=0$}       & \multirow{3}{*}{Vietnamese}     &Baseline       & 0.53 (0.09)   & 1.39 (0.41)   & 1.19 (0.37)   & 3.51 (0.31)   & 1.01 (0.67) \\
					&               &Post v3 static & 0.81 (0.08)   & 1.77 (0.37)   & 1.41 (0.27)   & 3.35 (0.36)   & 3.77 (0.33) \\
					&				&Post v3 dynamic        & 0.81 (0.08)   & 1.74 (0.35)   & 1.37 (0.24)   & 3.42 (0.36)   & 3.83 (0.33) \\
					&&&&&&&\\
					& \multirow{3}{*}{Japanese}     &Baseline       & 0.63 (0.09)   & 1.38 (0.32)   & 1.13 (0.19)   & 3.63 (0.30)   & 1.09 (0.60) \\
					&               &Post v3 static & 0.66 (0.09)   & 1.50 (0.25)   & 1.20 (0.12)   & 3.12 (0.34)   & 3.65 (0.44) \\
					&               &Post v3 dynamic        & 0.67 (0.09)   & 1.51 (0.26)   & 1.22 (0.13)   & 3.07 (0.30)   & 3.86 (0.28) \\
					&&&&&&&\\
					& \multirow{3}{*}{French}       &Baseline       & 0.68 (0.12)   & 1.40 (0.29)   & 1.10 (0.12)   & 3.81 (0.28)   & 1.34 (0.71) \\
					&               &Post v3 static & 0.72 (0.12)   & 1.59 (0.28)   & 1.22 (0.15)   & 3.29 (0.38)   & 3.76 (0.38) \\
					&               &Post v3 dynamic        & 0.73 (0.13)   & 1.59 (0.28)   & 1.23 (0.16)   & 3.24 (0.39)   & 3.85 (0.34) \\
					&&&&&&&\\
					& \multirow{3}{*}{Russian}      &Baseline       & 0.69 (0.10)   & 1.34 (0.24)   & 1.08 (0.08)   & 3.78 (0.27)   & 1.33 (0.67) \\
					&               &Post v3 static & 0.72 (0.11)   & 1.55 (0.26)   & 1.18 (0.13)   & 3.22 (0.40)   & 3.75 (0.34) \\
					&               &Post v3 dynamic        & 0.72 (0.12)   & 1.54 (0.28)   & 1.19 (0.15)   & 3.18 (0.39)   & 3.81 (0.30) \\
					&&&&&&&\\
					& \multirow{3}{*}{Germany}      &Baseline       & 0.74 (0.09)   & 1.46 (0.27)   & 1.12 (0.14)   & 3.83 (0.29)   & 1.30 (0.68) \\
					&               &Post v3 static & 0.79 (0.09)   & 1.56 (0.27)   & 1.21 (0.14)   & 3.37 (0.33)   & 3.88 (0.35) \\
					&               &Post v3 dynamic        & 0.79 (0.09)   & 1.53 (0.28)   & 1.20 (0.15)   & 3.34 (0.33)   & 3.98 (0.27) \\
					&&&&&&&\\
					& \multirow{3}{*}{Portuguese}   &Baseline       & 0.68 (0.10)   & 1.49 (0.36)   & 1.13 (0.22)   & 3.80 (0.25)   & 1.36 (0.71) \\
					&               &Post v3 static & 0.74 (0.08)   & 1.61 (0.27)   & 1.21 (0.14)   & 3.33 (0.36)   & 3.84 (0.31) \\
					&               &Post v3 dynamic        & 0.74 (0.08)   & 1.59 (0.26)   & 1.20 (0.13)   & 3.30 (0.36)   & 3.93 (0.28) \\
					&&&&&&&\\
					& \multirow{3}{*}{Chinese}      &Baseline       & 0.62 (0.12)   & 1.51 (0.36)   & 1.11 (0.16)   & 3.82 (0.26)   & 1.28 (0.68) \\
					&               &Post v3 static & 0.65 (0.12)   & 1.61 (0.27)   & 1.21 (0.11)   & 3.30 (0.36)   & 3.60 (0.42) \\
					&               &Post v3 dynamic        & 0.66 (0.12)   & 1.60 (0.27)   & 1.22 (0.11)   & 3.23 (0.33)   & 3.69 (0.37) \\
					\hline
				\end{tabular}
				\caption{So sánh các metrics của mô hình đề xuất với các ngôn ngữ khác nhau có $\name{SNR}=0$}
			\end{subtable}
		\end{sidewaystable}
	
		\clearpage
		
		\begin{sidewaystable} \ContinuedFloat
			\centering
			\begin{subtable}{\textwidth}
				\centering
				\begin{tabular}{llcccccc}
					\hline
					\multicolumn{1}{c}{\textbf{Bộ kiểm thử}}	& \multicolumn{1}{c}{\textbf{Ngôn ngữ}}	& \textbf{Mô hình}	& \textbf{STOI}	& \textbf{NBPESQ}	& \textbf{WBPESQ}	& \textbf{SIG}	& \textbf{BAK} \\
					\hline
					\multirow{27}{*}{$\name{SNR}=5$}       & \multirow{3}{*}{Vietnamese}     &Baseline       & 0.84 (0.08)   & 1.78 (0.41)   & 1.29 (0.28)   & 4.15 (0.20)   & 1.63 (0.76) \\
					&               &Post v3 static & 0.89 (0.07)   & 2.14 (0.38)   & 1.65 (0.32)   & 3.65 (0.34)   & 4.01 (0.26) \\
					&				&Post v3 dynamic        & 0.89 (0.07)   & 2.10 (0.37)   & 1.59 (0.29)   & 3.70 (0.35)   & 4.06 (0.27) \\
					&&&&&&&\\
					& \multirow{3}{*}{Japanese}     &Baseline       & 0.73 (0.08)   & 1.45 (0.32)   & 1.15 (0.17)   & 3.80 (0.27)   & 1.33 (0.67) \\
					&               &Post v3 static & 0.77 (0.07)   & 1.75 (0.29)   & 1.33 (0.17)   & 3.37 (0.27)   & 4.03 (0.31) \\
					&               &Post v3 dynamic        & 0.77 (0.07)   & 1.74 (0.28)   & 1.33 (0.17)   & 3.33 (0.29)   & 4.12 (0.22) \\
					&&&&&&&\\
					& \multirow{3}{*}{French}       &Baseline       & 0.78 (0.12)   & 1.56 (0.36)   & 1.18 (0.23)   & 4.05 (0.24)   & 1.70 (0.73) \\
					&               &Post v3 static & 0.81 (0.11)   & 1.89 (0.32)   & 1.37 (0.19)   & 3.63 (0.34)   & 4.07 (0.34) \\
					&               &Post v3 dynamic        & 0.81 (0.12)   & 1.86 (0.33)   & 1.36 (0.20)   & 3.60 (0.35)   & 4.11 (0.31) \\
					&&&&&&&\\
					& \multirow{3}{*}{Russian}      &Baseline       & 0.77 (0.10)   & 1.53 (0.36)   & 1.15 (0.21)   & 3.96 (0.30)   & 1.79 (0.77) \\
					&               &Post v3 static & 0.78 (0.12)   & 1.80 (0.34)   & 1.29 (0.20)   & 3.50 (0.45)   & 3.99 (0.33) \\
					&               &Post v3 dynamic        & 0.78 (0.13)   & 1.76 (0.34)   & 1.28 (0.19)   & 3.45 (0.47)   & 4.00 (0.32) \\
					&&&&&&&\\
					& \multirow{3}{*}{Germany}      &Baseline       & 0.83 (0.07)   & 1.65 (0.35)   & 1.22 (0.24)   & 4.07 (0.23)   & 1.69 (0.78) \\
					&               &Post v3 static & 0.86 (0.06)   & 1.88 (0.32)   & 1.36 (0.20)   & 3.71 (0.30)   & 4.17 (0.30) \\
					&               &Post v3 dynamic        & 0.86 (0.07)   & 1.84 (0.34)   & 1.35 (0.21)   & 3.69 (0.30)   & 4.23 (0.21) \\
					&&&&&&&\\
					& \multirow{3}{*}{Portuguese}   &Baseline       & 0.77 (0.08)   & 1.70 (0.36)   & 1.22 (0.22)   & 4.02 (0.23)   & 1.68 (0.70) \\
					&               &Post v3 static & 0.82 (0.07)   & 1.88 (0.29)   & 1.33 (0.17)   & 3.65 (0.32)   & 4.12 (0.29) \\
					&               &Post v3 dynamic        & 0.82 (0.07)   & 1.85 (0.30)   & 1.31 (0.17)   & 3.63 (0.32)   & 4.17 (0.24) \\
					&&&&&&&\\
					& \multirow{3}{*}{Chinese}      &Baseline       & 0.70 (0.12)   & 1.73 (0.40)   & 1.21 (0.20)   & 4.03 (0.22)   & 1.57 (0.68) \\
					&               &Post v3 static & 0.74 (0.12)   & 1.95 (0.33)   & 1.38 (0.20)   & 3.58 (0.31)   & 3.84 (0.37) \\
					&               &Post v3 dynamic        & 0.73 (0.12)   & 1.91 (0.33)   & 1.36 (0.19)   & 3.50 (0.30)   & 3.88 (0.34) \\
					\hline
				\end{tabular}
				\caption{So sánh các metrics của mô hình đề xuất với các ngôn ngữ khác nhau có $\name{SNR}=5$}
			\end{subtable}
		\end{sidewaystable}
	
		\clearpage
		
		\begin{sidewaystable} \ContinuedFloat
			\centering
			\begin{subtable}{\textwidth}
				\centering
				\begin{tabular}{llcccccc}
					\hline
					\multicolumn{1}{c}{\textbf{Bộ kiểm thử}}	& \multicolumn{1}{c}{\textbf{Ngôn ngữ}}	& \textbf{Mô hình}	& \textbf{STOI}	& \textbf{NBPESQ}	& \textbf{WBPESQ}	& \textbf{SIG}	& \textbf{BAK} \\
					\hline
					\multirow{27}{*}{$\name{SNR}=10$}       & \multirow{3}{*}{Vietnamese}     &Baseline       & 0.90 (0.07)   & 2.11 (0.46)   & 1.54 (0.37)   & 4.32 (0.15)   & 2.20 (0.69) \\
					&               &Post v3 static & 0.92 (0.06)   & 2.49 (0.42)   & 1.90 (0.39)   & 3.89 (0.32)   & 4.16 (0.24) \\
					&				&Post v3 dynamic        & 0.92 (0.06)   & 2.45 (0.40)   & 1.83 (0.36)   & 3.91 (0.33)   & 4.20 (0.24) \\
					&&&&&&&\\
					& \multirow{3}{*}{Japanese}     &Baseline       & 0.83 (0.07)   & 1.66 (0.42)   & 1.27 (0.31)   & 4.00 (0.21)   & 1.66 (0.69) \\
					&               &Post v3 static & 0.85 (0.05)   & 2.05 (0.34)   & 1.51 (0.22)   & 3.63 (0.25)   & 4.26 (0.23) \\
					&               &Post v3 dynamic        & 0.86 (0.06)   & 2.06 (0.38)   & 1.52 (0.27)   & 3.60 (0.26)   & 4.29 (0.19) \\
					&&&&&&&\\
					& \multirow{3}{*}{French}       &Baseline       & 0.86 (0.10)   & 1.89 (0.47)   & 1.37 (0.30)   & 4.23 (0.20)   & 2.08 (0.72) \\
					&               &Post v3 static & 0.88 (0.10)   & 2.28 (0.38)   & 1.59 (0.28)   & 3.89 (0.32)   & 4.27 (0.30) \\
					&               &Post v3 dynamic        & 0.87 (0.10)   & 2.25 (0.40)   & 1.58 (0.29)   & 3.88 (0.34)   & 4.29 (0.28) \\
					&&&&&&&\\
					& \multirow{3}{*}{Russian}      &Baseline       & 0.85 (0.09)   & 1.76 (0.45)   & 1.29 (0.29)   & 4.12 (0.31)   & 2.18 (0.75) \\
					&               &Post v3 static & 0.85 (0.11)   & 2.14 (0.42)   & 1.50 (0.29)   & 3.74 (0.47)   & 4.17 (0.31) \\
					&               &Post v3 dynamic        & 0.85 (0.11)   & 2.10 (0.42)   & 1.49 (0.29)   & 3.70 (0.49)   & 4.17 (0.33) \\
					&&&&&&&\\
					& \multirow{3}{*}{Germany}      &Baseline       & 0.90 (0.05)   & 1.96 (0.38)   & 1.40 (0.27)   & 4.23 (0.19)   & 2.12 (0.71) \\
					&               &Post v3 static & 0.91 (0.05)   & 2.24 (0.33)   & 1.56 (0.23)   & 3.90 (0.26)   & 4.32 (0.21) \\
					&               &Post v3 dynamic        & 0.91 (0.05)   & 2.20 (0.36)   & 1.54 (0.25)   & 3.89 (0.27)   & 4.33 (0.19) \\
					&&&&&&&\\
					& \multirow{3}{*}{Portuguese}   &Baseline       & 0.85 (0.08)   & 2.01 (0.46)   & 1.43 (0.34)   & 4.17 (0.21)   & 2.14 (0.70) \\
					&               &Post v3 static & 0.88 (0.07)   & 2.23 (0.34)   & 1.53 (0.26)   & 3.88 (0.28)   & 4.30 (0.22) \\
					&               &Post v3 dynamic        & 0.87 (0.07)   & 2.18 (0.33)   & 1.49 (0.24)   & 3.85 (0.29)   & 4.32 (0.19) \\
					&&&&&&&\\
					& \multirow{3}{*}{Chinese}      &Baseline       & 0.78 (0.13)   & 2.12 (0.53)   & 1.47 (0.41)   & 4.17 (0.19)   & 1.97 (0.79) \\
					&               &Post v3 static & 0.79 (0.13)   & 2.36 (0.35)   & 1.59 (0.25)   & 3.82 (0.27)   & 4.04 (0.35) \\
					&               &Post v3 dynamic        & 0.78 (0.13)   & 2.26 (0.37)   & 1.54 (0.25)   & 3.75 (0.28)   & 4.07 (0.32) \\
					\hline
				\end{tabular}
				\caption{So sánh các metrics của mô hình đề xuất với các ngôn ngữ khác nhau có $\name{SNR}=10$}
			\end{subtable}
		\end{sidewaystable}
	
		\clearpage
		
		\begin{sidewaystable} \ContinuedFloat
			\centering
			\begin{subtable}{\textwidth}
				\centering
				\begin{tabular}{llcccccc}
					\hline
					\multicolumn{1}{c}{\textbf{Bộ kiểm thử}}	& \multicolumn{1}{c}{\textbf{Ngôn ngữ}}	& \textbf{Mô hình}	& \textbf{STOI}	& \textbf{NBPESQ}	& \textbf{WBPESQ}	& \textbf{SIG}	& \textbf{BAK} \\
					\hline
					\multirow{27}{*}{$\name{SNR}=15$}       & \multirow{3}{*}{Vietnamese}     &Baseline       & 0.94 (0.05)   & 2.57 (0.50)   & 1.97 (0.48)   & 4.39 (0.15)   & 2.75 (0.60) \\
					&               &Post v3 static & 0.95 (0.05)   & 2.90 (0.44)   & 2.23 (0.45)   & 4.04 (0.30)   & 4.26 (0.22) \\
					&				&Post v3 dynamic        & 0.95 (0.05)   & 2.83 (0.41)   & 2.13 (0.41)   & 4.04 (0.32)   & 4.29 (0.21) \\
					&&&&&&&\\
					& \multirow{3}{*}{Japanese}     &Baseline       & 0.90 (0.05)   & 1.96 (0.48)   & 1.48 (0.41)   & 4.18 (0.15)   & 2.12 (0.69) \\
					&               &Post v3 static & 0.91 (0.04)   & 2.41 (0.33)   & 1.75 (0.26)   & 3.84 (0.23)   & 4.42 (0.14) \\
					&               &Post v3 dynamic        & 0.92 (0.04)   & 2.47 (0.36)   & 1.80 (0.31)   & 3.84 (0.24)   & 4.42 (0.14) \\
					&&&&&&&\\
					& \multirow{3}{*}{French}       &Baseline       & 0.91 (0.08)   & 2.27 (0.51)   & 1.69 (0.41)   & 4.33 (0.18)   & 2.59 (0.65) \\
					&               &Post v3 static & 0.92 (0.07)   & 2.64 (0.35)   & 1.84 (0.32)   & 4.08 (0.26)   & 4.42 (0.21) \\
					&               &Post v3 dynamic        & 0.91 (0.08)   & 2.59 (0.37)   & 1.81 (0.32)   & 4.07 (0.27)   & 4.42 (0.19) \\
					&&&&&&&\\
					& \multirow{3}{*}{Russian}      &Baseline       & 0.90 (0.07)   & 2.02 (0.51)   & 1.50 (0.39)   & 4.17 (0.36)   & 2.57 (0.65) \\
					&               &Post v3 static & 0.88 (0.11)   & 2.41 (0.48)   & 1.69 (0.37)   & 3.85 (0.51)   & 4.26 (0.34) \\
					&               &Post v3 dynamic        & 0.88 (0.11)   & 2.34 (0.47)   & 1.65 (0.34)   & 3.82 (0.53)   & 4.26 (0.33) \\
					&&&&&&&\\
					& \multirow{3}{*}{Germany}      &Baseline       & 0.94 (0.04)   & 2.37 (0.45)   & 1.77 (0.42)   & 4.33 (0.17)   & 2.68 (0.65) \\
					&               &Post v3 static & 0.94 (0.04)   & 2.65 (0.36)   & 1.86 (0.32)   & 4.08 (0.24)   & 4.44 (0.17) \\
					&               &Post v3 dynamic        & 0.94 (0.04)   & 2.64 (0.36)   & 1.85 (0.32)   & 4.08 (0.24)   & 4.43 (0.16) \\
					&&&&&&&\\
					& \multirow{3}{*}{Portuguese}   &Baseline       & 0.90 (0.07)   & 2.45 (0.55)   & 1.82 (0.49)   & 4.24 (0.21)   & 2.57 (0.65) \\
					&               &Post v3 static & 0.91 (0.07)   & 2.56 (0.34)   & 1.76 (0.29)   & 4.04 (0.28)   & 4.41 (0.19) \\
					&               &Post v3 dynamic        & 0.91 (0.07)   & 2.52 (0.33)   & 1.72 (0.28)   & 4.02 (0.29)   & 4.41 (0.17) \\
					&&&&&&&\\
					& \multirow{3}{*}{Chinese}      &Baseline       & 0.83 (0.13)   & 2.53 (0.51)   & 1.84 (0.42)   & 4.23 (0.17)   & 2.41 (0.69) \\
					&               &Post v3 static & 0.83 (0.13)   & 2.74 (0.35)   & 1.86 (0.32)   & 3.98 (0.24)   & 4.20 (0.31) \\
					&               &Post v3 dynamic        & 0.82 (0.12)   & 2.63 (0.37)   & 1.76 (0.31)   & 3.93 (0.25)   & 4.20 (0.28) \\
					\hline
				\end{tabular}
				\caption{So sánh các metrics của mô hình đề xuất với các ngôn ngữ khác nhau có $\name{SNR}=15$}
			\end{subtable}
		\end{sidewaystable}
	
		\clearpage
		
		\begin{sidewaystable} \ContinuedFloat
			\centering
			\begin{subtable}{\textwidth}
				\centering
				\begin{tabular}{llcccccc}
					\hline
					\multicolumn{1}{c}{\textbf{Bộ kiểm thử}}	& \multicolumn{1}{c}{\textbf{Ngôn ngữ}}	& \textbf{Mô hình}	& \textbf{STOI}	& \textbf{NBPESQ}	& \textbf{WBPESQ}	& \textbf{SIG}	& \textbf{BAK} \\
					\hline
					\multirow{27}{*}{$\name{SNR}=20$}       & \multirow{3}{*}{Vietnamese}     &Baseline       & 0.96 (0.04)   & 3.08 (0.48)   & 2.51 (0.51)   & 4.38 (0.19)   & 3.25 (0.46) \\
					&               &Post v3 static & 0.96 (0.05)   & 3.22 (0.47)   & 2.49 (0.51)   & 4.14 (0.30)   & 4.33 (0.21) \\
					&				&Post v3 dynamic        & 0.96 (0.05)   & 3.12 (0.44)   & 2.38 (0.48)   & 4.14 (0.30)   & 4.35 (0.20) \\
					&&&&&&&\\
					& \multirow{3}{*}{Japanese}     &Baseline       & 0.95 (0.03)   & 2.36 (0.55)   & 1.85 (0.55)   & 4.26 (0.14)   & 2.68 (0.57) \\
					&               &Post v3 static & 0.95 (0.02)   & 2.79 (0.31)   & 2.03 (0.31)   & 3.97 (0.21)   & 4.50 (0.14) \\
					&               &Post v3 dynamic        & 0.96 (0.02)   & 2.92 (0.34)   & 2.16 (0.37)   & 4.00 (0.21)   & 4.50 (0.12) \\
					&&&&&&&\\
					& \multirow{3}{*}{French}       &Baseline       & 0.95 (0.05)   & 2.68 (0.51)   & 2.13 (0.48)   & 4.35 (0.19)   & 3.09 (0.54) \\
					&               &Post v3 static & 0.94 (0.06)   & 2.95 (0.35)   & 2.11 (0.37)   & 4.19 (0.24)   & 4.49 (0.21) \\
					&               &Post v3 dynamic        & 0.94 (0.06)   & 2.92 (0.36)   & 2.06 (0.36)   & 4.19 (0.26)   & 4.48 (0.20) \\
					&&&&&&&\\
					& \multirow{3}{*}{Russian}      &Baseline       & 0.94 (0.07)   & 2.55 (0.59)   & 2.01 (0.58)   & 4.26 (0.31)   & 3.13 (0.50) \\
					&               &Post v3 static & 0.91 (0.09)   & 2.75 (0.49)   & 1.95 (0.48)   & 4.04 (0.46)   & 4.37 (0.30) \\
					&               &Post v3 dynamic        & 0.91 (0.09)   & 2.66 (0.47)   & 1.86 (0.44)   & 4.02 (0.45)   & 4.36 (0.29) \\
					&&&&&&&\\
					& \multirow{3}{*}{Germany}      &Baseline       & 0.97 (0.03)   & 2.76 (0.45)   & 2.21 (0.46)   & 4.35 (0.19)   & 3.11 (0.57) \\
					&               &Post v3 static & 0.96 (0.03)   & 2.98 (0.33)   & 2.14 (0.37)   & 4.17 (0.22)   & 4.50 (0.14) \\
					&               &Post v3 dynamic        & 0.96 (0.03)   & 2.99 (0.33)   & 2.16 (0.37)   & 4.17 (0.22)   & 4.48 (0.15) \\
					&&&&&&&\\
					& \multirow{3}{*}{Portuguese}   &Baseline       & 0.94 (0.05)   & 2.81 (0.54)   & 2.25 (0.53)   & 4.26 (0.20)   & 3.02 (0.52) \\
					&               &Post v3 static & 0.93 (0.06)   & 2.85 (0.34)   & 2.00 (0.34)   & 4.12 (0.24)   & 4.44 (0.18) \\
					&               &Post v3 dynamic        & 0.93 (0.06)   & 2.82 (0.36)   & 1.96 (0.35)   & 4.12 (0.26)   & 4.45 (0.17) \\
					&&&&&&&\\
					& \multirow{3}{*}{Chinese}      &Baseline       & 0.89 (0.10)   & 3.00 (0.49)   & 2.38 (0.47)   & 4.25 (0.15)   & 2.84 (0.62) \\
					&               &Post v3 static & 0.86 (0.10)   & 3.05 (0.35)   & 2.09 (0.37)   & 4.09 (0.22)   & 4.28 (0.28) \\
					&               &Post v3 dynamic        & 0.86 (0.10)   & 2.91 (0.38)   & 1.95 (0.35)   & 4.04 (0.24)   & 4.28 (0.26) \\
					\hline
				\end{tabular}
				\caption{So sánh các metrics của mô hình đề xuất với các ngôn ngữ khác nhau có $\name{SNR}=20$}
			\end{subtable}
		\end{sidewaystable}
	
	\clearpage

\section{Một số kết quả của ứng dụng lọc nhiễu}\label{section::results::app}

	Ứng dụng lọc nhiễu của chúng tôi được chia làm hai phần: \textit{Lọc nhiễu tĩnh} và \textit{Lọc nhiễu động}. Phần lọc nhiễu tĩnh được chúng tôi tích hợp vào ứng dụng, giao diện ứng dụng được chúng tôi thiết kế như trong \figref{re::main_app}. Trong \figref{re::main_app} thể hiện cửa sổ chính, cửa sổ điều chỉnh và cửa sổ ghi âm của ứng dụng chúng tôi.
	
		\begin{figure}[h]
			\centering
			\begin{subfigure}{.5\textwidth}
				\centering
				\includegraphics[width=70mm]{\reimg{ui_recording.jpg}}
				\caption{Cửa sổ ghi âm}
			\end{subfigure}%
			\begin{subfigure}{.5\textwidth}
				\centering
				\includegraphics[width=70mm]{\reimg{ui_conf.png}}
				\caption{Cửa sổ điều chỉnh}
			\end{subfigure}
		
			\begin{subfigure}{\textwidth}
				\centering
				\includegraphics[width=70mm]{\reimg{ui_main.PNG}}
				\caption{Cửa sổ chính}
			\end{subfigure}
			\caption{Một số hình ảnh của ứng dụng lọc nhiễu tĩnh}
			\label{re::main_app}
		\end{figure}
	
	Phần lọc nhiễu động được chúng tôi thiết kế như một dịch vụ của hệ điều hành và cho phép điều khiển thông qua một giao diện độc lập, các hình ảnh của ứng dụng này khi kích hoạt và đóng dịch vụ lọc nhiễu được cài đặt trên hệ điều hành được thể hiện như trong \figref{re::svc_app}.
	
		\begin{figure}[h]
			\centering
			\begin{subfigure}{.5\textwidth}
				\centering
				\includegraphics[width=60mm]{\reimg{ui_svc_on.PNG}}
				\caption{Dịch vụ đã khởi động}
			\end{subfigure}%
			\begin{subfigure}{.5\textwidth}
				\centering
				\includegraphics[width=60mm]{\reimg{ui_svc_off.PNG}}
				\caption{Dịch vụ đã đóng}
			\end{subfigure}
			\caption{Một số hình ảnh của ứng dụng lọc nhiễu động}
			\label{re::svc_app}
		\end{figure}
	
	Một trong những khó khăn lớn nhất của chúng tôi trong quá trình hoàn thành ứng dụng đó là việc kiểm thử ứng dụng lọc nhiễu động. Môi trường chạy ở dưới nền tảng của hệ điều hành được gọi là \textbf{môi trường không lỗi} (exception-free environment). Trong môi trường loại này, các lỗi phát sinh ra dù có nghiêm trọng cũng không thể bị ``bắt''(catch) bởi bất cứ cơ chế nào. Do đó, để tránh phát sinh tình trạng lỗi nghiêm trọng do lỗi của code ở kernel, chúng tôi cần kiểm tra rất kĩ các dòng này, kể cả trong quá trình viết code cũng như khi ra chạy thử. Đồng thời để kiểm tra sự ổn định của nó, chúng tôi cũng tiến hành kiểm thử trên từng đoạn năm phút hội thoại thực (dùng để đo \tableref{re::svc_latency_detail}). Vì vậy nên ở ứng dụng động sẽ không kiểm thử thực tế theo testcase mà sẽ được đem vào thử nghiệm trong thực tế.
	
	Ở ứng dụng tĩnh, ta sẽ sử dụng các trường hợp kiểm thử trên phần mềm sau cùng. Các trường hợp này được chia như sau
	
		\begin{enumerate}[1)]
			\item \enumentry{Bộ quản lý trạng thái:} Tạo liên tục một số lượng luồng (functional thread) bất kì (ít hơn 5 luồng), mỗi lần tạo cách nhau 0.1ms. Mỗi luồng chờ trong 5s và in ra thứ tự được tạo của mình. Nếu thứ tự thỏa mãn từ bé đến lớn thì bộ quản lý hoạt động đúng.
			\item \enumentry{Luồng:} Luồng kiểm thử được tạo ra, in ra timestamp mở đầu sau đó chờ trong 1s và sau đó lại in ra timestamp của thời điểm hiện tại. Nếu timestamp mở đầu bé hơn timestamp lúc sau thì luồng hoạt động đúng.
		\end{enumerate}
	
	Đây chính là hai thành phần cốt lõi của ứng dụng, các tác vụ khác hầu hết chỉ sử dụng thư viện để chạy mô hình, đọc dữ liệu từ microphone hoặc gán dữ liệu vào một biến ở môi trường. Một số tác vụ khác lại phải sử dụng bằng kiểm thực trực quan hóa như tác vụ hiển thị ra màn hình. Hầu hết việc kiểm thử của chúng tôi được thực hiện bằng thủ công và đây cũng chính là một trong những khó khăn lớn nhất trong luận văn này.
	